\documentclass[addpoints,12pt]{exam}

\usepackage{amsmath}
\usepackage{amsthm}

\printanswers %Remove to hide answers
\pagestyle{headandfoot} %Adds Headers and Footers
\runningheadrule
\firstpageheader{Math 151}{Exam 1}{July 4, 1776}
\runningheader{Math 151}
{Exam 1, Page \thepage\ of \numpages}
{July 4, 1776}
\firstpagefooter{}{\thepage}{}
\runningfooter{}{\thepage}{}


\begin{document}

%The box at the top, and the name
\begin{center}
\fbox{\fbox{\parbox{5.5in}{\centering
Answer the questions in the spaces provided on the
question sheets. If you run out of room for an answer,
continue on the back of the page.}}}
\end{center}
\vspace{0.1in}
\makebox[\textwidth]{Name:\enspace\hrulefill}
\vspace{0.2in}

%Question Formatting
%\qformat{\textbf{Question \thequestion}\quad (\thepoints)\hfill}
%Point Table


%\begin{center}
%\gradetable[h][questions]
%\end{center}

%Beginning Questions
\begin{questions}
	\question Solve the following linear equation 
   \[
 2(x-1)+3 = x-3(x-1)~.
\]

 \begin{solution}

 \end{solution}
 
\question Solve the following equation 
   \[
 \frac{x}{5}- \frac{1}{2} = \frac{x}{6}~.
\]

\question The length of an American football field is 200 feet more than its width. If the perimeter is 1040 feet, then how wide is the field? 
\question You have 150 dollars to inverst. Part of the money is invested in an account through Yankton Financial paying $15\%$ annual interest. The rest of the money is to be invested in a second account at Vermillian Bank paying $13\%$ interest. If you would like $2\$$ a year in interest, how much should you invest into each account? Feel free to keep your answer unsimplified. 

\question Perform the following computation using complex numbers. 
   \[
		 (7+2i)-(5-7i)= 
\]



\question Perform the following computation using complex numbers. 
\[
		 (3+5i)(-5-i)=
\]



\end{questions}

\end{document}
